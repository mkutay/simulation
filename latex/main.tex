\documentclass[12pt, a4paper]{scrartcl}
\renewcommand{\baselinestretch}{1.15}

\usepackage{changepage}
\usepackage{multicol}
\usepackage{paralist}
\usepackage[lmargin=0.9in, rmargin=0.9in, tmargin=1in, bmargin=1.3in]{geometry}

\setkomafont{disposition}{\normalfont\bfseries}
\parskip=8pt

\begin{document}

\pagenumbering{gobble}

\begin{titlepage}
  \begin{center}
    \LARGE
      \textbf{KING'S COLLEGE LONDON}

      \vspace{2cm}

      \begin{adjustwidth}{-1cm}{-1cm}
        \centering
        \Large
        \textbf{4CCS1PPA PROGRAMMING PRACTICE AND APPLICATIONS}
      \end{adjustwidth}
      
      \vspace{0.5cm}

      \Large
      \textbf{Third "Simulation" Coursework (Feb 2025)}

      \vspace{2cm}

      \Large
      Project Name: The Simulation 

      \vspace{1cm}

      \Large
      \begin{tabular}{l l}
        Student Name: & Mehmet Kutay Bozkurt \\
        Student ID: & 23162628 \\
        \vspace{0.5cm} & \\
        Student Name: & Anas Ahmed \\
        Student ID: & xxxxxxxx
      \end{tabular}
  \end{center}
  % \tableofcontents
\end{titlepage}

\begin{multicols}{2}

\pagenumbering{arabic}

\section{Introduction}

\noindent This simulation project integrates multiple components to create a dynamic ecosystem that a researcher can use to
study predator-prey-plant interactions, with the ability to add or remove any species or environmental factors easily
by editing the given JSON file.
The simulation is designed to simulate without any grid-based restrictions, allowing entities to move freely in a
continuous space. Specifically, the coordinates of the entities are stored as doubles from zero to the width and height
of the field.

\noindent The simulation smoothly runs at 60 frames per second by utilising a \verb|QuadTree| for entity searching and
collision detection. In each simulation "step," every entity is "updated" by calling its \verb|update()| method, which
handles movement, reproduction, and hunger. Entities make decisions based on other entities in their vicinity
(that is, located inside their \verb|sightRadius|) and the current state of the environment. Additionally, there is a
method to handle overcrowding in the simulation, for limiting the exponentioal growth of entities.

\section{Tasks Lists and Implementation Details}

\subsection{Base Tasks}

  \noindent \textbf{Diverse Species:} With how the simulation is implemented, adding new species is as easy as adding 
  the species' behavioural data into a JSON file. Each entity species can be of type \verb|Prey|, \verb|Predator|,
  or \verb|Plant|.
  
  \noindent \textbf{Two Predators Competes for the Same Food Source:}
  Since it is easy to add multiple species of predators, currently there are two predators, \verb|Wolf| and \verb|Fox|,
  that compete for the same food source, \verb|Rabbit|.
  
  \noindent \textbf{Distinguishing Gender:} Each entity has their own gender, represented in their genetics,
  which affects reproduction mechanics. Only entities of opposite genders can reproduce.
  
  \noindent \textbf{Tracking Time of Day:} A simulation clock governs the day/night cycle impacting entity behaviour.

\subsection{Challenge Tasks}

  \noindent \textbf{Adding Plants:} Plants have been added, featuring growth and reproduction dynamics.
  
  \noindent \textbf{Adding Weather:} Weather conditions and time of day influence behavior and visibility,
  creating a \textit{realistic} simulation environment.
  
  \noindent \textbf{Disease Dynamics:} 

\section{Code Quality Considerations}

\subsection{Coupling}


\subsection{Cohesion}


\subsection{Responsibility-Driven Design}


\subsection{Maintainability}


\section{Final Remarks}


\end{multicols}

\end{document}